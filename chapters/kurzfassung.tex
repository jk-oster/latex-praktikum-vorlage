\chapter{Kurzfassung}
\label{cha:Kurzfassung}

\paragraph{Umfang der Kurzfassung: ca.\ 200 Worte.}

\paragraph{Zum allgemeinen Inhalt des Berichts: \emph{Dieser Bericht beschreibt den Ablauf des Praktikums}, die \textbf{Aufgaben und durchgeführten Projekte} und Erfahrungen. Die eigenen Aktivitäten (Projekte) stehen dabei natürlich im Mittelpunkt und bilden den Hauptteil des Berichts. Wenn viele Kleinprojekte bearbeitet wurden, sollten einige davon exemplarisch genauer beschrieben werden. Neben der eigentlichen Arbeit sollten aber auch folgende weitere Aspekte berücksichtigt werden:}
%
\begin{itemize}
	\item Abläufe (Workflows) innerhalb des Unternehmens bzw.\ in Projekten
	(grafische Darstellungen können dabei nützlich sein),
	\item Arbeits- und Führungsstil, Kommunikation innerhalb des Unternehmens,
	\item Kommunikation nach außen (Kunden, Partner),
	\item Zeitsituation, Terminprobleme,
	\item Einbettung in das Team, soziale Erfahrungen,
	\item Einsatz von speziellen Techniken, Methoden und Werkzeugen,
	\item wichtige Herausforderungen oder Schwierigkeiten,
	\item Anforderungen in Bezug auf die Ausbildung im Studium (gut
	einsetzbare Kenntnisse, Defizite).
\end{itemize}
%
Die nachfolgenden Kapitelüberschriften sollen nur zur Orientierung für die
Struktur des Berichts dienen, über die konkrete Einteilung und den Wortlaut
kann man natürlich selbst entscheiden.

% Bilder
\begin{figure}[htbp] 
% !h -> Bild hier im Text erzwingen
% !b -> Bild am Ende der Seite erzwingen
% !t -> Bild am Anfang der Seite erzwingen
% htbp -> Versuche Bild hier einzufügen
% wenn nicht möglich am Anfang der Seite
% wenn nicht möglich am Ende der Seite
% wenn nicht möglich eigene Seite
    \centering
    \includegraphics[width=.2\textwidth]{logo}
    \caption{Das ist eine Abbildung}
    \label{fig:pic1}
\end{figure}